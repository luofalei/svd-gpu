\section{Bisection \& Twisted Algorithm}
The singular value decomposition of an arbitrary matrix $A\in R^{mn} (m>n)$ consists of two steps:
The first step is to reduce the initial matrix $A$ to bidiagonal form using Householder transformation.
The second step is to reduce the bidiagonal matrix into diagonal matrix.

All algorithms have to make use of Householder transformation to reduces the time cost of SVD.
The time complexity of QR algorithm, Jacobi algorithm and bisection and inverse algorithm will change from $O(n^3)$ to $O(n^4)$ without Householder transformation, while divide-and-conquer algorithm cannot work any more.
Thus, in this paper, we only focus on the second step with bisection and twisted algorithm.

The bisection and twisted algorithm is separated into two phases:
\begin{enumerate}
\item Obtain the singular values of bi-diagonal matrix by using bisection approach.
\item Obtain the conresponding left and right singular vectors of each singular values by using twisted factorization.
\end{enumerate}
We will illustrate these two phases in the following subsections.

\subsection{Bisection Algorithm}
Bisection algorithm is widely used in many application to find the root of a sophisticated equation which can not be solved directly.
It repeatedly bisects an interval and then selects a subinterval for further processing until convergence.
The algorithm is an approximate approach to solve the sophisticated equations.
It can obtain relative accuracy solution when the toleration is relative strict.

Suppose $B$ is an upper bidiagonal $n*n$ matrix with elements $b_{i,j}$ reduced by Householder transform.
The matrix $T = B^T B - \mu^2 I$ can be decomposed as
\begin{equation}
\label{eq:T}
T = B^T B - \mu^2 I = L D L^T
\end{equation}
where $D$ is $diag(d_{1}, \cdots, d_{n})$,
%\[ D =  \left( \begin{array}{cccc}
%d_{1} & 0     & \cdots & 0 \\
%0     & d_{2} & \cdots & 0 \\
%\vdots& \vdots& \ddots & \vdots \\
%    0 & 0     & \cdots & d_{n} \end{array} \right),
\begin{equation}
 L =  \left( \begin{array}{ccccc}
     1&      &       &        &  \\
 l_{1}& 1    &       & 0      &  \\
      & l_{2}& \ddots&        &  \\
      & 0    & \ddots& \ddots &  \\
      &      &       & l_{n-1}& 1
\end{array} \right) 
\label{eq:l}
\end{equation}
%$L$ are left bidiagonal matrices with diagonal elements 1s and sub-diagonal elements $l_{i}$.
Substitute $L$ and $D$ into Equation \ref{eq:T}, we can obtain the following equations
\begin{equation}
\left \{
\begin{aligned}
b_{1,1}^2 - \mu^2 &= d_1\\
b_{k-1,k-1} b_{k-1,k} &= d_{k-1} l_{k-1}\\
b_{k-1,k}^2 + b_{k,k}^2 - \mu^2 &= l_{k-1}^2 d_{k-1} + d_k
\end{aligned}
\right .
\label{eq:ldl}
\end{equation}
where $k = 2,3,\cdots,n$.
We define a temperary value $t_{k}$
\begin{equation}
t_{k} = t_{k-1} * (b_{k-1,k}^2 / d) - \mu^2.
\label{eq:tmp}
\end{equation}
After equivalent transformation, the Equations \ref{eq:ldl} can be rewrited as
\begin{equation}
\label{eq:negcount}
d_k = b_{k,k}^2 + t_{k}
\end{equation}

It is clear that matrix $D$ and matrix $T$ are two congruent symmetric matrices.
According to the Sylvester's law of inertia, matrix $D$ and matrix $T$ has have the same numbers of positive, negative, and zero eigenvalues.
We define $NegCount(\mu)$ function to be the number of negative eigenvalues in matrix $T$ with $\mu^2$ shift.
Since matrix $B$ and matrix $B^T$ have the same singular values,
$NegCount(\mu)$ is also the number of the singular values of $B$ which are less than $\mu$.
It is clear that if the floating point arithmetic is monotonic, then $NegCount(x)$ is a monotonically increasing function of $x$\cite{95ETNAbisecion}.
The $NegCount$ function is in Algorithm \ref{alg:negcount}.
%\alglanguage{pseudocode}
\begin{algorithm}
%\small
\caption{NegCount in Bisection Algorithm}
\label{alg:negcount}
\begin{algorithmic}[1]
\Procedure{$\mathbf{NegCount}$}{$n, B, \mu$}
  \State $d=1$, $t=0$, $cnt=0$, $b_{0,1}=0$;
  \For {$k = 1 \to n$}
    \State $t = t * (b_{k-1,k}^2 / d) - \mu^2$;
    \State $d = b_{k,k}^2 + t$;
    \If {$d < 0$}
      \State $cnt++$;
    \EndIf
  \EndFor
  \State \Return $cnt$;
\EndProcedure
\end{algorithmic}
\end{algorithm}

%Algorithm \ref{alg:negcount} is used to count the number of singular values of matrix $B$ those are less than $\mu$.
%From the algorithm, the number of singular values in an arbitrary interval $[\mu_1,\mu_2)$ are $c_{\mu_2} - c_{\mu_1}$.


%\alglanguage{pseudocode}
\begin{algorithm}
%\small
\caption{Bisection Algorithm}
\label{alg:bisection}
\begin{algorithmic}[1]
\Procedure{$\mathbf{Bisection}$}{$val, n, B, l, u, n_l, n_u,\tau$}
  \If {$n_l \geq n_r$ or $l > u$}
    \State No singular values in $[l,u)$;
  \EndIf
  \State Enqueue ($l, u, n_l, n_u$) to Worklist;
  \While {Worklist is not empty}
    \State Dequeue ($a, b, n_a, n_b$) from Worklist;
    \If {$b - a < \tau$}
      \For {$i=n_a+1 \to n_b$}
        \State $val[i] = \min(\max((a+b)/2,a),b)$;
      \EndFor
    \Else
      \State $m = MidPoint(a,b)$;
      \State NegCount($n_m, B, m$);
      \State $n_m = \min(\max(n_m,n_a),n_b)$;
      \If {$n_m > m_a$}
        \State Enqueue ($a, m, n_a, n_m$) to Worklist;
      \EndIf
      \If {$n_m < m_b$}
        \State Enqueue ($m, b, n_m, n_b$) to Worklist;
      \EndIf
    \EndIf
  \EndWhile
\EndProcedure
\end{algorithmic}
\end{algorithm}

Algorithm \ref{alg:bisection} is the serial bisection algorithm.
It is able to calculate the singular values in interval $[l,u)$, whose $NegCount$ are $n_l$ and $n_u$, seperately.
The basic steps of the algorithm for singular value are as follows:
\begin{enumerate}
\item Divides one interval containing singular values into two subintervals.
\item Utilizes $NegCount$ algorithm to get the number of singular values in subintervals.
\item Selects the subinterval(s) which contain singular values for further bisection.
\item Repeated 1-3 until all subintervals becomes convergence.% which means the left border and right border of the subintervals are almost equal to each other.
\item The singular values are considered as the midpoint of the subintervals.
\end{enumerate}

The algorithm provides theoretical basis to calculate the singular values in a specified interval.
The interval containing all the singular values can be calculated by Gershgorin circle theorem.
%In the algorithm, $MidPoint$ should avoid the possibility of overflow or underflow.

\subsection{Twisted Algorithm}
After the first step of obtaining the singular values, it is still necessary to get their corresponding left and right singular vectors.
The simplest method to obtain the singular vectors is the power method, which can find only the largest singular value and the corresponding singular vector\cite{97bookalgebra}.
The second method is inverse iteration.
However, there is no guarantee that the singular vectors are accurate or orthogonal.
Additionally, the inverse iteration requires $O(n^3)$ to obtain all the singular vectors.

We utilize the improved twisted factorization to calculate the singular vectors.
The algorithm improves the accurate or orthogonal problems in general bisection and inversed algorithm.
It can also solve the singular vectors whose singular values are clustered together.

Suppose $\lambda$ is one singular value of bidiagonal matrix $B$.
Then the matrix $B^T B - \lambda^2 I$ can be decomposed as
\begin{equation}
B^T B - \lambda^2 I = L D_L L^T = U D_U U^T
\end{equation}
where $D_L=diag(\alpha_1, \cdots, \alpha_n)$, $D_U = diag(\beta_1, \cdots, \beta_n)$, $L$ is the same form with Equation \ref{eq:l}
%\[ L =  \left( \begin{array}{ccccc}
%     1&      &       &        &  \\
% l_{1}& 1    &       & 0      &  \\
%      & l_{2}& \ddots&        &  \\
%      & 0    & \ddots& \ddots &  \\
%      &      &       & l_{n-1}& 1
%\end{array} \right),
\begin{equation}
 U =  \left( \begin{array}{ccccc}
  1 & u_{1}&       &       &  \\
    & 1    & u_{2} & 0     &  \\
    &      & 1     & \ddots&  \\
    & 0    &       & \ddots& u_{n-1} \\
    &      &       &       & 1
 \end{array} \right)
\end{equation}
%$L$ is left bidiagonal matrix, $U$ is upper bidiagonal matrix.
%The diagonal elements in $L$ and $U$ are 1s.
%The subdiagonal elements are $l_k$ in $L$, and the superdiagonal elements are $u_k$ in $U$, separately.

Given the $LDL^T$ and $UDU^T$ decomposition, we consider the twisted factorization of the shifted matrix
\begin{equation}
B^T B - \lambda^2 I = N_k D_k N_k^T
\end{equation}
where $N_k$ is the twisted matrix.
$k$ is the index of the minimum $\gamma$ in Eq. \ref{eq:gamma}.
%\begin{equation}
%N_k =  \left( \begin{array}{cccccccc}
%     1&       &        &  & \cdots & 0 \\
% l_{1}& \ddots&        &  & 0 \\
%      & \ddots& 1      &  & 0 \\
%      &       & l_{k-2}& 1 & \vdots \\
%      &       &        & \eta_k & 1 \\
%      &       &        & \cdots & 1 \\
%      &       & 0      & \cdots & 1 \\
%      &       &        & \cdots & 1 \end{array} \right)
%\end{equation}
\begin{equation}
\label{eq:gamma}
k = \arg \min_{1\le i \le n} \gamma_{i} =
\begin{cases}
\beta_1 & \text{if } i=1 \\
\beta_i - \alpha_i * l_{i-1} & \text{if } 2\le i\le n-1\\
\alpha_n & \text{if } i=n
\end{cases}
\end{equation}

Thus, the corresponding singular vector of $\lambda$ is solving the following matrix equations and normalize the solution.
\begin{equation}
\label{eq:unnorm}
N_k^T z_k = e_k
\end{equation}
where $e$ is the unitary matrix, and $z$ is the non-normalization solution of singular vectors.

Our algorithm has a small modification when singular values are clustered together.
Suppose $r$ is the multiplicity of singular values of matrix $B$. 
The algorithm selects the indices of the first $r$ minimum value of $gamma$.
Each index $k$ has one different twisted factorization, and thus has a different Eq. \ref{eq:unnorm} to solve.
The singular vectors are also orthogonal to each others\cite{09NLAAtwisted}.
%\alglanguage{pseudocode}
\begin{algorithm}
%\small
\caption{Twisted Factorization}
\label{alg:twisted}
\begin{algorithmic}[1]
\Procedure{$\mathbf{Twisted}$}{$q, n, B, \mu$}
  \LineComment $l$ is the number of different singular values, $l\le n$;
  \For {$i = 1 \to l$}
    \State Computes the matrix $S = B - \mu_i^2 I$;
    \State Computes the $LDL'$ decomposition $S = LD_LL'$;
    \State Computes the $UDU'$ decomposition $S = UD_UU'$;
    \State Computes $gamma$ based on Eq \ref{eq:gamma};
    \State Find the multiplicity $m$ of singular value $\mu$;
    \State Find $m$-th minimum $k = min_j  |\gamma(j)|$;
    \For {each k}
    \State $z_k = 1$, $z_{k-1} = -L_{k-1,k}$, $z_{k+1} = -U_{k,k+1}$;
    \For {$j = k+2 \to n$}
      \State $z_j = -U_{j-1,j}*z_{j-1}$
    \EndFor
    \For {$j = k-2 \to 1$}
      \State $z_j = -L_{j+1,j}*z_{j+1}]$
    \EndFor
    \State Scale vector $q = z/||z||_2$;
    \EndFor
  \EndFor
\EndProcedure
\end{algorithmic}
\end{algorithm}

Algorithm \ref{alg:twisted} is the algorithm to obtain the corresponding singular vectors of given singular values in serial.
The cost for every singular vector transformation is $O(n)$, and the total cost of the transformations is $O(n^2)$.

