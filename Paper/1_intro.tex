\section{introduction}
Singular value decomposition (SVD) has been widely used in various fields in recent years,
such as noise reduction in signal processing,
low-rank approximations in linear algebra, 
objects classification in computer vision,
latent semantic indexing in information retrieval,
supervised and unsupervised algorithms in machine learning,
data compression in information theory.
Current SVD approaches could solve small-scale data in acceptable time and space requirements.
Unfortunately, with the advent of the era of data explosion, small-scale data processing has been difficult to meet the needs of big data.
Thus, it is still a problem to extend the SVD algorithms on a very large data set.

The SVD can be broken down into two steps\cite{65SIAM}:
The first step is to reduce the initial matrix to bidiagonal form using Householder transformations.
The second step is to diagonalize the resulting matrix using bidiagonal SVD algorithms.
Most of researchers focus on the second step with iteration methods\cite{58iter1,90iter2,65iter3}.
This is because the first step executes only one time by Householder transform,
while the second step executes much more than once dependent on accuracy requirement.

Three algorithms has been introduced to solve the bidiagonal SVD.
QR algorithm is recognized as a powerful and effective method.
However, it is only the fastest algorithm for small matrices whose sizes are less than 25\cite{97bookalgebra}.
The complexity of $O(n^3)$ flops will give the execution time a rapid increment, as matrix size becomes large.
Also, the iteration times still rise to a large integer when matrix size is large.
Jacobi algorithm is the most accuracy method in practise.
However, the $O(n^3)$ flops with big constant will cause the algorithm much more slower than other algorithms.
Also, the iteration time of Jacobi algorithm is much larger than those of QR algorithm.

Divide-and-conquer algorithm is assumed as the fastest method of SVD when matrices are large\cite{94DCSVD}.
It takes only $O(n^{2.3})$ flops on average\cite{97bookalgebra}.
But the singular values are not relative accuracy when merging, let alone singular vectors.
If all the singular values are distributed in the worst case, the time cost will increase to $O(n^3)$.

In addition to the speed and relative accuracy, all of these three algorithms above have two common disadvantage for parallel computing:
\begin{enumerate}
\item These algorithms have more or less data dependence.
The heavy data dependence makes SVD algorithm not suitable for parallelization and architecture extension.
\item  These algorithms require $O(n^2)$ memory locations to save temporary variables.
The large memory locations needed in these algorithms will not be able to calculate the SVD when the matrix is large enough.
\end{enumerate}

Most of SVD applications, such as principle component analysis (PCA), need only a small subset of the singular values and sigular vectors.
However, the algorithms above are not able to calculate the subset directly.
The only way is to select a part from all singular values and vectors.

Fortunately, bisection and inverse (BI) algorithm could find the subset easily.
Bisection and Inverse iteration takes $O(nk)$ flops to find $k$ singular values and singular vectors, and $O(nk^2)$ in the worst case of $k$ singular values are clustered.
It is much faster than other algorithms when only a small subset of singular values and vectors are needed in a huge data size.
But the inverse iteration does not guarantee the accuracy and orthogonality of the computed singular vectors in the case of singular values clustered.

In this paper, we present a new SVD approach, bisection and twisted (BT) algorithm.
It inherits the advantages of BI algorithms.
It has proved that the singular vectors are accurate and orthogonal in twisted algorithm\cite{09NLAAtwisted}.
Comparing to other algorithms, BT approach only requires $O(n^2)$ flops to obtain all singular values and singular vectors\cite{09NLAAtwisted,05UCB}, even in the worst case.
It is faster than any other algorithms metioned above.
The algorithm does not have any data dependence, which is excellent for parallelism on multi-threads and extention to multi-GPU.
The algorithm can also obtain one singular value and its conresponding vector in $O(n)$ flops.
Additionally, the algorithm needs only $O(kn)$ memory location to store temporary memory.
It is good to extend to many threads or many cores.

The rest of the paper is organized as follows.
Section 2 discusses the related work.
A high-level serial algorithm is given in Section 3.
Section 4 describes the implementation of our bisection and twisted algorithm on GPU.
Section 5 presents the experimental results and profiling data of GPU kernels.
Future work and conclusion are given in Section 6.

